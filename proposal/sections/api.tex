\subsection{API Overview}

Our current running name for the library is \texttt{\APIName}\footnote{We still need a proper acronym for it, e.g. \APIshort}, i.e. the next ship that will lead the flotilla after yet another sunk boat. We propose to implement the following interface in \APIName:


\begin{itemize}
	\item Server side:
    \begin{itemize}
    \begin{ttfamily}
      \item initServer(name)
      \item onReceive(msg, callback)
      \item commitState(callback)
    \end{ttfamily}
    \end{itemize}
    \item Client side:
    	\begin{itemize}
    	\begin{ttfamily}
    		\item initClient()
            \item connect(serverName)
            \item send(msg, payload)
    	\end{ttfamily}
    	\end{itemize}
\end{itemize}


We briefly discuss the major functions of our proposed API in the following subsections. Throughout the discussion, we use the TA queue example to illustrate our API usage.

{\bf Server initialization and service instantiation: } the first functions to initiate a server are {\ttfamily initServer} and {\ttfamily onReceive}. The former starts the local server in the device's browser and, after initialization, assigns a host name for that server. The later register entry points for services offered by that server.

In our queue system, one would initialize a server and define two functions to handle requests to enqueue students and also to dequeue them once they are helped. Additionally, the server provides the queue service in order to provide the current state of the queue. If no recognizable service is requested, the TAQueue server responds with the queue service.

\begin{lstlisting}[language=JavaScript]
    function getQueue(req, event) { ... }
    function handleEnqueue(req, event) { ... }
    function handleDequeue(req, event) { ... }

    (function main(){
        server = sship.initServer("TAQueue");
        server.onReceive('queue', getQueue, 
            default=true);
        server.onReceive('enqueue', handleEnqueue);
        server.onReceive('dequeue', handleDequeue);
    })();
\end{lstlisting}

{\bf Establishing connections: } as a server starts running, it broadcasts its name in the local-area network and clients in the same network can discover this server. A client device needs a single line of code to initialize itself. Upon initialization it will lookup for host servers in the network. Once a list of servers is retrieved and displayed, a client may select a server to connect to. In our TA queue example, we explicitly know the server name and skip the server list phase:

\begin{lstlisting}[language=JavaScript]
    (function main(){
        client = sship.initClient().connect("TAQueue");
    })();
\end{lstlisting}


{\bf Data exchange: } clients can ask for services through the {\ttfamily send} function. The function explicitly takes a requested service as one of its parameters and a payload as its second one. In our queue system, two distinct clients may request to enqueue themselves.

\begin{lstlisting}[language=JavaScript]
    (function main(){
        client1 = sship.initClient().connect("TAQueue");
        client1.send(enqueue, 
            {student: "Arthur", csid: "cs4321"});

        client2 = sship.initClient().connect("TAQueue");
        client2.send(enqueue, 
            {student: "Paul", csid: "cs9876"});
    })();
\end{lstlisting}

{\bf Updating the server state: } Finally, it is necessary to define which data structures or variables are important for a server, thus the {\ttfamily commitState} function receives a function which is executed every time that a service is successfully requested and executed in that server. Revisiting our {\ttfamily server = sship.initServer("TAQueue")} code snippet, we would add a final function to define how the server would be updated after queueing/dequeuing students.

\begin{lstlisting}[language=JavaScript]
    var queue = [];
    function currentQueue() { return queue; }

    (function main(){
        server = sship.initServer("TAQueue");
        ...
        server.commitState(currentQueue);
    })();
\end{lstlisting}

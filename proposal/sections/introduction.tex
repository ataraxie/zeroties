\section{Introduction}
\label{sec:introduction}


% Introduce client and server model over the Internet
The client-server model describes a scenario in which one or many clients require a service or resources from a centralized server. Tacit in this model is the assumption that the server has access to resources that are unavailable to the client, whether this be compute power, storage, sensitive data, etc. 


% Introduce devices with lower computational power that can work as servers for some applications
Regarding the growing domain of the Internet of things (IoT), the proliferation of ``smart'' devices with lower constraints on computational power, storage, or sensitive data, opened new doors to a set of applications that relax some of the assumptions in client-server model. For instance, a queue application used by a TA during office hours: the TA spins up the application on her phone, and as students enter the room, they connect to the server on the TA's phone and request to be enqueued. In this example, however, we need to be explicit about the gains obtained from adhering to this client-server model, rather than implementing this application as a peer-to-peer application.


% Introduce zeroconf networks and device/service discovery
In the aforementioned example, we assume that {\it (1)} devices are in a local-area network and {\it (2)} client-devices are able to discover the server-device and its offered service in that network. Such assumptions may be satisfied in a local-area network using existing IP-based routing. Nonetheless, adding and configuring devices in a local-area network may be error prone, time consuming, or ill suited for the users of these devices. Therefore automatic approaches to identify devices [and their services] in a local-area network, such as Zero-configuration networking, are desired. 


% Introduce fly web
The suite of protocols in Zero-configuration networking (mDNS/ DNS-SD)~\cite{rfc6762,rfc6763} provides {\it (i)} the automatic assignment of IP addresses and host naming (mDNS), and {\it (ii)} service discovery (DNS-SD). Once a device has an assigned name/IP and other devices can discover this device provided services, they can use application layer protocols and communicate with each other. As an example, the Mozilla Firefox\footnote{\url{https://www.mozilla.org/en-US/foundation/}} FlyWeb\footnote{\url{https://wiki.mozilla.org/FlyWeb}} extension leverages this suite of protocols to allow clients of Web applications to start their own local Web server from within the browser. This server is then advertised in the local network such that other devices in the network can detect the new server and can connect to it via the browser. 


% Discuss fault tolerance
The FlyWeb extension caught our attention as it offers a range of new possibilities for Web applications and local vs. global network behavior. However, its current implementation has a severe limitation, the complete lack of fault tolerance. When the local server dies, the client-server network dies with it. Since this technology is inherently driven by the idea of any device being able to become a server, we assume that it is more likely that servers misbehave in comparison to the  ``traditional'' server model. 



We therefore regard fault tolerance as very important in such networks and we think that the lack of mechanism for graceful recovery from server abruptly disconnections is problematic.
For instance, in our queue example, maybe the TA needs to leave the room momentarily, and therefore has to leave the local network. In this event, we wish for the entire application state, and even the ability of new students to enqueue as they arrive, to persist even as the initial server host leaves the network (``distributed failover''). When the TA returns, the application seamlessly returns to being hosted on her device (``failback''). 


% State our project proposal
Considering the aforementioned discussion, {\bf we propose an approach to add fault tolerance to FlyWeb}. We hypothesize that a technology like FlyWeb is a good basis for fault tolerance through replication, since any participant can become the server. In the situation of a failing server, we think it is intuitive that a client can become the ``next" server and all other clients establish a connection to the new server. Obviously, replication comes at the cost of complexity. We intend to analyze the different replication strategies and choose one that adheres best to our scenario. We aim to deliver our approach in a JavaScript library -- named {\texttt{\APIName{}}} -- that provides fault tolerance to FlyWeb application developers without exposing the underlying technical details.



% Final considerations:: I'm not sure if we need this paragraph
As a final remark, it is important to emphasize that it is neither our intent to shoe-horning the client-server application model into working as a peer-to-peer service nor assume that all web applications can use our proposed approach. But we imagine some concrete use cases, such as the TA queue system, that would benefit from this fault tolerant behavior.
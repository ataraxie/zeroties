\section{Related Work}
\label{sec:related_work}

\textbf{Publish/Subscribe} is a widespread messaging pattern that offers a rendezvous style channel for senders (publishers) and receivers (subscribers). 
Using this pattern, participants do not need to be aware of what other specific participants are in the system and no direct communication channels between sender and receiver are required.
This provides better scalability and more flexible network topology.
The publish/subscribe pattern has been elaborated largely in both the distributed systems~\cite{eugster_2003, banavar_1999, eugster_2000} and software engineering~\cite{carzaniga_1998, cugola_2001, filho_2005} communities.

\textbf{Zeroconf} has been investigated by several researchers both in terms of how the protocols can be improved~\cite{gunes_2002, bohnekamp_2003, jara_2012} and what problems they impose on networks~\cite{stolikj_2006, hong_2007}. The most widely available implementation of Zeroconf is Apple's \textit{Bonjour}~\footnote{https://support.apple.com/en-ca/bonjour, accessed 2019-04-17} software that is also available for Windows and Linux. There are also Linux-~\footnote{https://www.avahi.org/, accessed 2019-04-17} and similar Windows-specific~\footnote{http://techgenix.com/overview-link-local-multicast-name-resolution/, accessed 2019-04-17} implementations. These implementations are widely used by different applications to communicate in ad-hoc networks, including home entertainment (e.g. Google Chromecast~\footnote{https://store.google.com/product/chromecast, accessed 2019-04-17}, Amazon Fire TV~\footnote{https://amazonfiretv.blog, accessed 2019-04-17}), desktop applications (e.g. Adobe Photoshop~\footnote{https://www.adobe.com/products/photoshop.html, accessed 2019-04-17}, iTunes~\footnote{https://www.apple.com/ca/itunes/, accessed 2019-04-17}).

\textbf{Zeroties.} In contrast to these platform-specific implementations and applications of Zeroconf, we are interested in providing a framework for web application developers to use the compelling ad-hoc networking characteristics of Zeroconf.
Our approach provides this framework on top of the publish/subscribe pattern makes the compelling Zeroconf properties available to a wide variety of web applications in an efficient manner.
To the best of our knowledge, Mozilla FlyWeb~\footnote{https://wiki.mozilla.org/FlyWeb, accessed 2019-04-17} has been the only attempt that shared our motivation of Zeroconf in-browser web servers.
Unfortunately it had major problems (see Section~\ref{sec:background_and_motivation}) and has been abandoned by Mozilla.
Zeroties not only extracts the functionality from Mozilla FlyWeb but also abstracts the implementation away from platform-specific details and fixes a range of issues in Mozilla's now-abandoned addon.
In particular, it extends the compatibility from Firefox-only to Firefox and Chrome~\footnote{Zeroties is also easily adaptable for other browsers' addon ecosystems.}, and improves performance of Zeroconf communication at least by a factor of two.
\section{Limitations and Future Work}
\label{sec:limitations_and_future_work}

We are generally very content with the substantial improvements we achieved with Zeroties for Zeroconf web applications and especially our Successorships framework.
Nevertheless, a range of limitations and potentials for improvements remains.
Also, with this project we have just targeted one subsection of the \textit{Limitations and Future Work} section in the Successorships paper: \textit{Platform dependency and time-to-recovery.}
We can definitely see research opportunities in this field.
With Zeroties in particular, we see limitations in our \textit{evaluation}, \textit{scalability}, \textit{deployability}, \textit{usability and network requirements}, and \textit{imposed load} on the network.

\textbf{Threats to our evaluation.}
The performance measurements we performed for Zeroties were run on one machine with multiple browser windows representing different nodes.
We had intended to run our scenario with multiple machines, just as we did in the Successorships evaluation, which we had run on a small Wi-Fi network created with a mobile hotspot.
Unfortunately, we were not able to set up such a network such that the Zeroconf protocols were supported.
Neither did the Wi-Fi of our workplace support these.
We have found that this is due to special security restrictions and we could verify that our system works on a regular home Wi-Fi.
However, we were not prepared for these difficulties and had to fall back on a single-machine evaluation due to time constraints.
Despite all this, we argue that our evaluation serves its purpose as initial proof for significant performance improvement with Zeroties compared to the old FlyWeb version: after all, we ran our scenario on one machine with both versions and the FlyWeb version showed performance comparable to the numbers from the evaluation of Successorships.

\textbf{Scalability.}
We have noticed degradation in performance with higher number of nodes in our experiments: the system becomes notably slower with more than approx. 10 clients joining the network.
Additionally, we have observed that inconsistencies occurred using the Firefox version of the Zeroties addon after approx. 10 clients joining the network and sending a significant amount of messages.
This was also the reason for performing our performance measurements only with Google Chrome.
While, further evaluation with Firefox after fixing these issues would harden our results, we do not think our sole focus on Chrome for now constitutes a significant threat.
But these issues certainly need some inspection and debugging before Zeroties can be deployed for larger and more critical applications.

\textbf{Deployability.}
What we have implemented as the \textit{Zeroties daemon} was originally intended to reside in the browser addon.
We could not find a way to achieve this due to the security restrictions imposed on addons by browsers (i.e. they can start neither a HTTP server nor a WebSocket server).
This is why Zeroties users will now have to start the OS-level (i.e. Node.js application) daemon to make use of our system.
We intend to research further possibilities to go without this separate dependency in the future.
For example, there may be a more advanced possibility for security configuration in browsers that we were not able to find yet.

\textbf{Usability and network requirements.}
Different characteristics of Zeroties in its current version may distract users (or even make our system entirely unusable for them). 
First, we are using hard-coded ports for our HTTP and WebSocket servers.
While one might think initially that this could simply be solved by providing configuration options, it is actually not that simple: depending on the number of applications, Zeroties may need multiple HTTP and WebSocket servers that must be started and stopped dynamically.
Nevertheless, such configuration options are required because, as obvious, hard-coded port numbers are not feasible in practice.
Another problem in terms of usability are the URLs of Zeroties servives that remain IP-based.
This has already been a limitation in Successorships and we could not yet overcome this problem.
Finally, we have not yet implemented filtering of DNS-SD services to only include Zeroties services in the lists of available services.
In practice, this means that, for example, a Google Chromecast service may currently appear in the Zeroties menu.
This is a simple fix using an internal prefix for Zeroties service keys and respective filtering of all DNS-SD services.

\textbf{Network load.}
Each Zeroties daemon currently polls the network router for a list of available Zeroties services.
While we see this approach mostly as justified in our specific target environments (see Section~\ref{sub:design}), it has been hard for us to estimate how much load our system will impose on the local network.
Currently we have configured our daemon to poll the service list in intervals of one second.
With each host participating in a Zeroties network performing this polling strategy, there is a risk of significant impact on the network.
We intend to evaluate this in more detail with a future evaluation of Zeroties.





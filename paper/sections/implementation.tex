\section{Implementation}
\label{sec:implementation}

\subsection{Languages and Platforms}
\label{sub:languages_and_platforms}

All components are written in JavaScript. 
The Zeroties application is written using Node.js\footnote{https://nodejs.org/en/, accessed 2019-04-17}, the browser addons are written with the vendor-specific addon frameworks.
We decided for JavaScript for all components due to several reasons:
\begin{itemize}
\item We wanted the OS-level Zeroties application be as platform-independent as possible. 
Node.js is a stable environment on all major operating systems by now.
\item We wanted to lean on a library implementing the low-level Zeroconf protocols.
We found a capable, mature, and well-maintained library, \textit{dnssd}\footnote{https://www.npmjs.com/package/dnssd, accessed 2019-04-17} available for this purpose on Node.js.
\item Browser addons must be implemented in JavaScript.
Despite the fact that we could not integrate the core Zeroties functionality into our browser addons (see Section~\ref{sec:approach}) we still wanted to avoid  introducing a language barrier that would prevent us to do so even without browser-specific restrictions.
\item Starting a HTTP server and a Websocket server in Node.js and connecting to these servers from an addon is inherently straight-forward.
This functionality was required for our purpose.
\end{itemize}

\subsection{Zeroties Application}
\label{sub:zeroties_application}

what it is:
- the core Zeroties app
- any application wishing to use Zeroties must communicate with this app

what does it do:
- facilitates publishing services on the local network
- notifies clients on updates to available services list
- handles client connections and forwards to =

how does it work:
- services list is maintained/updated via mdns/dnssd
- client apps connect to the Zeroties app via a Websocket connection
- on connecting, clients are automatically subscribed to recieve updates about changes to the list of services
- 

could it be improved?:


\subsubsection{API}

\subsection{Browser Addons}
\label{sub:browser_addons}

TODO: Chris

\subsubsection{API}

\subsection{Communication and Control Flow in Zeroties}
\label{sub:communication}

TODO: Chris
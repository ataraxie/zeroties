\documentclass[sigconf]{acmart}
\usepackage[utf8]{inputenc}

\settopmatter{printacmref=false}
\renewcommand\footnotetextcopyrightpermission[1]{} % removes footnote with conference information in first column
\pagestyle{plain}

\usepackage{url}
\usepackage{hyperref}
\hypersetup{
    colorlinks=true,
    linkcolor=blue,
    filecolor=magenta,      
    urlcolor=cyan,
}
\urlstyle{same}


\usepackage{listings}
\usepackage{lipsum}
\usepackage{tabularx}
\usepackage{makecell}
\usepackage{pbox}
\usepackage{caption}
\captionsetup[table]{skip=5pt}


% Enabling Javascript syntax highlight in code snippet - BEGIN 
% https://tex.stackexchange.com/questions/89574/language-option-supported-in-listings
\usepackage{color}
\definecolor{lightgray}{rgb}{.9,.9,.9}
\definecolor{darkgray}{rgb}{.4,.4,.4}
\definecolor{purple}{rgb}{0.65, 0.12, 0.82}
\definecolor{darkgreen}{rgb}{0, .64, 0}

\lstdefinelanguage{JavaScript}{
  keywords={typeof, new, true, false, catch, function, return, null, catch, switch, var, if, in, while, do, else, case, break},
  keywordstyle=\color{blue}\bfseries,
  ndkeywords={class, export, boolean, throw, implements, import, this},
  ndkeywordstyle=\color{darkgray}\bfseries,
  identifierstyle=\color{black},
  sensitive=false,
  comment=[l]{//},
  morecomment=[s]{/*}{*/},
  commentstyle=\color{purple}\^amily,
  stringstyle=\color{darkgreen}\ttfamily,
  morestring=[b]',
  morestring=[b]"
}

\lstset{
   extendedchars=true,
   basicstyle=\footnotesize\ttfamily,
   showstringspaces=false,
   showspaces=false,
   tabsize=2,
   breaklines=true,
   showtabs=false,
   captionpos=b,
   frame=single,
   xleftmargin=0.5em,
   belowcaptionskip=0em
}

\newcommand{\mybox}[1]{\vspace{5pt}\noindent\fbox{\parbox{0.97\columnwidth}{#1}}\vspace{5pt}}

\setlength{\belowcaptionskip}{-1em}
% Enabling Javascript syntax highlight in code snippet - END

\newcommand{\APIName}{Successorships }
\newcommand{\APINameNoSpace}{Successorships}

\newcommand{\APIshort}{Shippy }
\newcommand{\rpm}{\raisebox{.2ex}{$\scriptstyle\pm$}}

\newcommand{\accessed}{accessed 2017-12-11}

\usepackage{array}
\newcolumntype{L}[1]{>{\raggedright\let\newline\\\arraybackslash\hspace{0pt}}m{#1}}
\newcolumntype{C}[1]{>{\centering\let\newline\\\arraybackslash\hspace{0pt}}m{#1}}
\newcolumntype{R}[1]{>{\raggedleft\let\newline\\\arraybackslash\hspace{0pt}}m{#1}}


\graphicspath{{figures/}{pictures/}{images/}{./}}

%opening
\title{Zeroties: a Publish/Subscribe Service for Applications in Local Ad-Hoc Networks}

\author{Chris Satterfield \qquad Felix Grund}
\affiliation{
    \institution{University of British Columbia}
    \city{Vancouver} 
    \state{BC} 
  }

\begin{document}

\begin{abstract}
Despite the Internet having become the most commonly used infrastructure for communication over the last two decades, there is also a set of applications that communicate via local area ad-hoc networks. 
While these are mostly scoped to dedicated domains like home entertainment and Wi-Fi printers, it is likely that this trend will grow with the advancements of ``smart'' devices and the ``Internet of Things''. 
In a previous project, we built a framework for developing fault-tolerant web applications based on such local ad-hoc networks and the Zeroconf protocol suite. 
A dependency to an obsolete Mozilla component made this framework slow and platform-dependent and made adoption in practice highly unlikely. 
To overcome this, we present Zeroties, an asynchronous publish-subscribe service for web applications operating in local ad-hoc networks. 
With a mixed-method evaluation we show that Zeroties effectively makes our framework platform-independent and recover from failures nearly 10 times faster than the previous version.
\end{abstract}

\maketitle

\section{Introduction}
\label{sec:introduction}

Throughout the last decade we have seen the Internet become the most commonly used infrastructure for communication: regardless of physical location, people and their devices communicate via messengers and VoIP, and collaborate via live editing tools like, for example, Google Docs, Sheets, and, Slides. 
At the same time, we have seen some applications communicating over local area networks become increasingly common for certain scenarios like home entertainment (e.g. Google Chromecast, Apple Bonjour, Spotify Connect) or Wi-Fi printers. 
With the advancements of ``smart'' devices and the ``Internet of Things'' (IoT) it is likely that this trend will grow beyond these currently still narrowly scoped application domains.

The adoption of standards that eliminate the burden of manual configuration of network devices further contributes to this movement. One such standard that has received widespread usage is Zero-configuration networking~\cite{guttman_2001} and its protocols mDNS~\cite{cheshire_2013_mdns} for service advertisement and DNS-SD for service discovery~\cite{cheshire_2013_dnssd}.
Using the \textit{Zeroconf} protocols, devices can publish named services in the local network and discover such services automatically in an ad-hoc fashion.
While most applications for Zeroconf networks are shipped with specific hardware (e.g. Google Chromecast dongle) there have recently been attempts to provide software environments for developers to enable them write their own applications on already existing hardware infrastructure.
One such application was Mozilla FlyWeb\footnote{https://wiki.mozilla.org/FlyWeb}, an addon for the Firefox browser that made it possible to advertise and discover services from within Web applications through a JavaScript API.
In a previous project, we created \textit{Successorships}~\footnote{https://github.com/ataraxie/successorships}, a JavaScript library exposing an easy-to-use API to build fault-tolerant Zeroconf web applications.
Successorships was built on top of Mozilla Flyweb as one main part of its architecture.
This decision confronted us with significant problems:
\begin{itemize}
    \item FlyWeb had been declared abandoned by Mozilla even before we finished our work on our library.
    \item The implementation of the Zeroconf protocols in FlyWeb was slow to a degree that made our library fairly unusable in practice.
    \item FlyWeb contained a bug that made our library usable on MacOS only.
\end{itemize}

To overcome our troubles in Successorships, we introduce Zeroties, a platform-independent asynchronous publish/subscribe service for Zeroconf advertisement and discovery.
We carefully reviewed different publish/subscribe designs~\cite{eugster_2003} and implemented a communication scheme based on \textit{asynchronous notifications}.
The operations exposed by Zeroties are as follows:
\begin{itemize}
    \item \textbf{Publish}: publish a Zeroties service and advertise it in the local network
    \item \textbf{Subscribe}: listen for updates on the list of available Zeroties services
\end{itemize}

Zeroties ships with two components: (1) a standalone OS-level application, and (2) addons for Chrome and Firefox that connect to this application.
With our addon implementations for Chrome and Firefox we aim to show that our approach translates well between different browsers and does not share the restrictions of FlyWeb.
Our browser addons expose an API to web applications that comprises the full service/discovery functionality of Zeroconf.
As a result, we have successfully eliminated the ties that prevented Successorships from usage in practice.

To evaluate Zeroties, we first created the webapp-based presentation for this project using Successorships in combination with the Zeroties daemon and the addon for Google Chrome.
Running the presentation on Chrome proved that we successfully broke ties with FlyWeb and Mozilla Firefox.
Furthermore, we could see in this example application that recovery from server failures was significantly faster than with the previous version based on FlyWeb.
To evaluate these findings empirically, we repeated the performance measurements from the Successorships project, both with the previous version based on FlyWeb and the new version based on Zeroties.
We simulated XXX server failures and subsequent recoveries in a small mobile network and found that the system recovered in average about three times as fast with Zeroties than with FlyWeb.

In summary, we make the following contributions:
\begin{itemize}
    \item Zeroties, a asynchronous standalone publish/subscribe service for Zeroconf applications.
    \item Addons for Chrome and Firefox that make this service available to web applications.
    \item An empirical evaluation based on a Zeroties sample application indicating significant performance improvements.
\end{itemize}

The remainder of this paper is organized as follows: Section~\ref{sec:background_and_motivation} provides some background and motivation on why the idea for Zeroties came to be. Section~\ref{sec:approach} presents the system model and design goals of Zeroties, before Section~\ref{sec:implementation} describes its implementation. We evaluate Zeroties in Section~\ref{sec:evaluation} and suggest limitations and future work in Section~\ref{sec:limitations_and_future_work}. Section~\ref{sec:related_work} situates our work in the context of related research and Section~\ref{sec:conclusion} concludes the paper.




\section{Background and Motivation}
\label{sec:background_and_motivation}

We have been compelled by the idea to make the Zeroconf protocols available to web applications.
For example, consider the scenario of a group of people, each with their own device, collaborating on a Google Doc document.
Every keystroke in this document will need to be sent to a Google web server and then be ``pushed'' too all other group members.
Depending on geological location, this pattern of communication can impose roundtrips around the globe and lead to significant delays, despite the physical proximity of the devices.
The fact that not everybody around the globe is equipped with the newest network infrastructure makes this even more proplematic.

Zeroconf web applications provide a solution to this problem.
In the described Google Doc scenario, one person in the group would access the the Google Doc on the Internet and subsequently become a server in the local network.
Her device would then serve the resources it fetched from the actual web server to local clients who maintain a channel to this local server.
Effectively, only one Internet connection is required for the whole group, rather than one connection for every device\footnote{Obviously, this also has the advantage that only one device needs to have Internet connection in the first place. This is can be very interesting especially in scenarios where only dedicated participants in a network have such a connection, for example due to security restrictions.}, because all clients except the locally serving client access the original application (in this case Google Docs). Figure~\ref{fig:architecture_shift} illustrates this shift in architecture.

\begin{figure}[h]
    \centering
    \includegraphics[width=\columnwidth]{architecture_shift}
    \caption{Architecture shift: on the left side is the traditional web application architecture where all devices are connected to a web server through the Internet. On the right is the Zeroconf architecture where only one client requires such a connection and becomes a server to all other clients in the network.}
    \label{fig:architecture_shift}
\end{figure}

Evidently, this pattern of communication has one major problem: it has a single point of failure, namely the client acting as a local server. 
Solving this problem was our motivation behind Successorships; whenever the currently serving client failed, a new client in the network was elected as the new server and connectivity in the network was re-established automatically.
Successorship thus provided a framework to build fault-tolerant Zeroconf web applications with an easy-to-use JavaScript API.
However, our decision to built it on Mozilla FlyWeb proved to be a major limitation.
Not only did we limit usage of our library to one specific browser vendor; we found out during the course of the project that Mozilla had already deprecated FlyWeb in favor of other priorities.
To make things worse, FlyWeb had a dependency to a Firefox core component (the module responsible for launching a web server within the browser) that was only shipped in a range of versions of the developer edition of Firefox.
In addition to this platform dependency that would, in essence, prevent our framework to be used in practice, the implementation of the Zeroconf protols in FlyWeb was incredibly slow.
According to our evaluation, recovery from failure of the currently serving client took more than 30 seconds in many cases.
With a continuously growing 'Limitations and Future Work' section, we were finally struck by the discovery of an unresolved issue in FlyWeb that made our framework only work on MacOS.
Despite our thoughtful and eager motivations, we had built a tool that nobody would ever use in practice.
Hence the call to get rid of this dependency entirely and provide our own layer below  Successorships.


\section{Approach}
\label{sec:approach}

\subsection{Environment and System Architecture}
\label{sub:architecture}

Figure~\ref{fig:architecture} illustrates the environment, placement of Zeroties, and the communication between components.
We describe the parts in the Figure by refering to the numbers in the ellipsis.

\begin{figure}[h]
    \centering
    \includegraphics[keepaspectratio,width=6cm]{architecture}
    \caption{System architecture and environment in Zeroties.}
    \label{fig:architecture}
\end{figure}

\begin{enumerate}
\item \textbf{Router}~\footnote{Note that we use the term router somewhat casually in this context. Our notion of router is basically any part in the local network that serves the list of available Zeroconf services, as it can be obtained with DNS-SD. This may involve DHCP, IPv6 Router Advertisement Options or other mechanisms as specified on page 27f of~\cite{cheshire_2013_dnssd}. In other words, we simply call the component that stores the service list obtained from a DNS-SD-capable router.}: we use the protocols provided by Zeroconf (\cite{cheshire_2013_dnssd, cheshire_2013_mdns}) for Zeroties. 
In essence, this means that an authoritative list of currently available Zeroties services (as a subset of all Zeroconf services) can be obtained from the router at any given time.

\item \textbf{Hosts}: there can be an arbitrary number of hosts in our local network. A host can be any device in the network with an instance of Zeroties running (i.e. any device with an OS capable of Zeroconf/DNS-SD protocols).

\item \textbf{Zeroties Daemon}: the instance of Zeroties running on this device. 
This is an OS-level background application maintaining a list of services.
This list is a mirror of the service list obtained from the router by means of the Zeroconf protocols. 
The list of services constitutes the state of the distributed system\footnote{Note that there is a clear line between Successorships and Zeroties: Zeroties does not know about the arbitrarily complex state of Successorships or any other app that builds on Zeroties.}.
We originally intended to integrate this application into browser addons, but were restricted by browser policies.
In particular, there is no possibility to start a web server or publish a Zeroconf service from within a browser addon.
Hence our alternative solution of an OS-level app that communicates with our browser addons through stable bi-directional message channels.

\item \textbf{Browser Addons}: we require browser addons that mediate between web applications and the Zeroties daemon. These addons expose a JavaScript API to apps built on top of Zeroties.
We have implemented addons for Google Chrome and Mozilla Firefox that constitute such a mediator and we have run Successorships apps with these addons (see Section~\ref{sec:evaluation}).
These addons use Zeroties as a publish/subscribe service.
Furthermore, we have built a UI in form of a browser menu button and a popup that shows the currently available list of services.
This UI is refreshed on-the-fly and mirrors the list of services in Zeroties.
Note that the Zeroties daemon itself is not restricted to these addons; they only provide a useful abstraction to web applications running in browsers.
Different applications can utilize the services provided by the Zeroties daemon without these components.

\item \textbf{Publish/Subscribe operations}: applications built on top of Zeroties interact with the Zeroties daemon using the common publish/subscribe operations.
\textit{Publish} is the publishing of a Zeroties service and \textit{subscribe} makes the application receive changes to the list of available Zeroties services. 

\item \textbf{Advertisements and service discovery}: the Zeroties daemon maintains a list of Zeroties services that is obtained from the router.
In a sense, Zeroties acts as a middleware between the router and applications built on top of Zeroties (applications using Successorships as framework being one example). 
Whenever a Zeroties application publishes a service using the Zeroties publish/subscribe API, this will result in the publishing of a service in the network (service advertisement). 
The second part of this communication channel is service discovery.
We describe both service advertisement and service discovery in more detail in Section~\ref{sec:implementation}.
\end{enumerate}


\subsection{Design Decisions and Goals}
\label{sub:design}

Zeroties is based on system models as in the example described in Section~\ref{sec:background_and_motivation}: local ad-hoc network applications. We claim that there is a dedicated set of such applications, for example:
\begin{itemize}
\item a project presentation at a meetup where the audience can connect to the presentation and interact with it.
\item an application for printer control in an office.
\item an application for the heating system of a hotel.
\item multiplayer mode for browser-based games.
\end{itemize}

Such applications have in common a limited number of nodes (generally < 100) and comparably lax requirements for time-to-recovery from system failures: certain downtimes can be tolerated as long as a consistent state is reached eventually.
For example, a running application converging to a consistent state within a time frame of multiple seconds after failures is acceptable.
Our particular goals behind Zeroties are described in the following paragraphs.

\textbf{Publish/subscribe strategy}. 
We reviewed the different publish/subscribe strategies described in~\cite{eugster_2003} and decided to focus on an asynchronous invocation/callback style communication pattern as described in~\cite{eugster_2003}§3.3. 
Asynchrony is essential for the communication between the Zeroties daemon and its applications.
For example, applications built on Zeroties should not be blocked while waiting for the successful advertisement of a service, but rather perform this action in the background while the application remains responsive in the foreground.

\textbf{Consistency guarantees}.
The shared state in Zeroties systems is defined by (1) the list of currently available Zeroties services, and (2) the application-defined state that is shared between hosts in the network.
If an instance is not up-to-date with this state, this can have different consequences. 
First, as long as applications are not notified that a new service has been created, that service remains unavailable.
Second, a service that has failed but remains in the list can result in an error upon connecting to that service. 
Third, if notifications about updates of the application-defined state are delayed between hosts, this can have significant consequences.
However, given our described system model of local ad-hoc network applications, we postulate that it is sufficient for most use cases if hosts converge to a consistent state \textit{eventually} and therefore decided for \textit{optimistic replication}~\cite{saito_2005}.
The paper defines \textit{eventual consistency} as ``a weak guarantee [that] is enough for many optimistic replication applications, but some systems provide stronger guarantees, e.g., that a replica's state is never more than one hour old''.
Due to our requirements for an asynchronous publish/subscribe pattern and our awarenes that, according to the CAP theorem~\cite{gilbert_2012} we cannot achieve both high consistency and availability, our work can be considered as trading consistency in favor of high availability.
Nevertheless, we consider the time frames for reaching consensus in our previous project (> 30 seconds in many cases) as insufficient and we want our new system to do so within a time frame of \textit{5 seconds}.

\textbf{Fault tolerance}:
In the case of Zeroties, fault tolerance is directly connected to consistency guarantees.
Take Successorships as an example for a Zeroties application and assume the scenario of a failing server~\footnote{In the Successorships example, by \textit{server} we refer to the \textit{the currently serving client}. Remember that in our world any client of the web application can become a server.}.
The service that was offered by the failed server will be removed from the list of available Zeroties services and the change will be propagated to Zeroties applications that poll information from the router (see Section~\ref{sub:architecture}).
The application can then decide how to deal with this change.
In the Successorships example, this will involve the selection of a new server (and therefore a new Zeroties service) and other clients connecting to it.
In that sense, the system will have recovered from the failure of the server (and therefore have converged to a consistent state), as soon as (1) the failed service was removed from the list, (2) the newly elected service was added to the list, and (3) all changes propagated to all clients in the system.
Consequently, Zeroties enables its applications to recover from failures in the same time frame as the system reaches consensus.

\textbf{Performance}:
As described earlier, the performance of our previous implementation of Zeroconf based on FlyWeb showed significant bottlenecks and proved insufficient for practical usage.
With Zeroties we had a few concrete goals in mind regarding time frames. 
First, the \textit{publishing of services} and \textit{notifications about new services} from the perspective of a Zeroties application should take \textit{less than 5 seconds}.
\textit{Communication between Zeroties and its applications} should be in the range of \textit{milliseconds}.

\textbf{Reliability}:
Our system should remain highly reliable as long as our DNS-SD communication scheme between Zeroties and the router is robust and the communication channels between Zeroties and its applications are stable.

\textbf{Polling strategy:}
We decided on a polling strategy for obtaining the list of services from the network router using DNS-SD.
The list of available services is updated based on changes detected between the last version of the services list in the Zeroties daemon and the most recently polled services list.
Clearly, this polling strategy will impose traffic on the local network, with all Zeroties hosts polling the router for the services list.
However, we argue that frequent changes in the available services list and the presumed low number of nodes in our system model justifies this approach.
A different strategy which could allow for a larger number of nodes without the potential problem of overloading the router with requests would be to implement this communication with a dedicated Zeroties leader and followers, for example using Paxos~\cite{lamport_2001} or Raft~\cite{ongaro_2014}.
However, this would mostly be a transformation of communication between Zeroties hosts and the router towards host-host communication. 
More importantly, we would waive the advantage of the direct use of the authoritative list of services maintained by the network.
Nevertheless, we are aware that a design with the leader-and-followers approach, despite adding significant complexity, would have advantages as soon as Zeroties networks reach a certain size.




\section{Implementation}
\label{sec:implementation}

\subsection{Languages and Platforms}
\label{sub:languages_and_platforms}

All components are written in JavaScript. 
The Zeroties application is written using Node.js\footnote{https://nodejs.org/en/, accessed 2019-04-17}, the browser addons are written with the vendor-specific addon frameworks.
We decided for JavaScript for all components due to several reasons:
\begin{itemize}
\item We wanted the OS-level Zeroties application be as platform-independent as possible. 
Node.js is a stable environment on all major operating systems by now.
\item We wanted to lean on a library implementing the low-level Zeroconf protocols.
We found a capable, mature, and well-maintained library, \textit{dnssd}\footnote{https://www.npmjs.com/package/dnssd, accessed 2019-04-17} available for this purpose on Node.js.
\item Browser addons must be implemented in JavaScript.
Despite the fact that we could not integrate the core Zeroties functionality into our browser addons (see Section~\ref{sec:approach}) we still wanted to avoid  introducing a language barrier that would prevent us to do so even without browser-specific restrictions.
\item Starting a HTTP server and a Websocket server in Node.js and connecting to these servers from an addon is inherently straight-forward.
This functionality was required for our purpose.
\end{itemize}

\subsection{Zeroties Application}
\label{sub:zeroties_application}

what it is:
- the core Zeroties app
- any application wishing to use Zeroties must communicate with this app

what does it do:
- facilitates publishing services on the local network
- notifies clients on updates to available services list
- handles client connections and forwards to =

how does it work:
- services list is maintained/updated via mdns/dnssd
- client apps connect to the Zeroties app via a Websocket connection
- on connecting, clients are automatically subscribed to recieve updates about changes to the list of services
- 

could it be improved?:


\subsubsection{API}

\subsection{Browser Addons}
\label{sub:browser_addons}

TODO: Chris

\subsubsection{API}

\subsection{Communication and Control Flow in Zeroties}
\label{sub:communication}

TODO: Chris

\section{Evaluation}
\label{sec:evaluation}

\subsection{Sample Successorships Application}

In our previous project, we created a web application using the presentation framework \textit{Reveal.js}~\footnote{https://github.com/hakimel/reveal.js/, accessed 2019-04-17} to present Successorships to graduate students of the computer science department of the University of British Columbia.
The application used the Successorships API that made use of the Zeroconf protocols based on FlyWeb.
Obviously, the presenters had to have a Mac computer and a dedicated old version of Firefox Developer Edition with the FlyWeb addon installed to be able to present.
Moreover, the previously mentioned performance problems were clearly visible, with recovery of the system from intentionally injected faults taking over 20 seconds at times.
In a similar fashion, we presented Zeroties to an audience of graduate students at the same university.
The presentation technicalities remained the same, except that the API usage within Successorships was changed from FlyWeb to Zeroties.
To demonstrate this new independence of Firefox, the presentation was performed on Google Chrome.
A significant decrease in time-to-recovery could also clearly be observed.

\subsection{Empirical Evaluation}

\section{Limitations and Future Work}
\label{sec:limitations_and_future_work}

We are generally very content with the substantial improvements we achieved with Zeroties for Zeroconf web applications and especially our Successorships framework.
Nevertheless, a range of limitations and potentials for improvements remains.
Also, with this project we have just targeted one subsection of the \textit{Limitations and Future Work} section in the Successorships paper\footnote{XXX}: \textit{Platform dependency and time-to-recovery.}
We can definitely see research opportunities in this field.
With Zeroties in particular, we see limitations in our \textit{evaluation}, \textit{scalability}, \textit{deployability}, \textit{usability and network requirements}, and \textit{imposed load} on the network.

\textbf{Threats to our evaluation.}
The performance measurements we performed for Zeroties were run on one machine with multiple browser windows representing different nodes.
We had intended to run our scenario with multiple machines, just as we did in the Successorships evaluation, which we had run on a small Wi-Fi network created with a mobile hotspot.
Unfortunately, we were not able to set up such a network such that the Zeroconf protocols were supported.
Neither did the Wi-Fi of our workplace support these.
We have found that this is due to special security restrictions and we could verify that our system works on a regular home Wi-Fi.
However, we were not prepared for these difficulties and had to fall back on a single-machine evaluation due to time constraints.
Despite all this, we argue that our evaluation serves its purpose as initial proof for significant performance improvement with Zeroties compared to the old FlyWeb version: after all, we ran our scenario on one machine with both versions and the FlyWeb version showed performance comparable to the numbers from the evaluation of Successorships.

\textbf{Scalability.}
We have noticed degradation in performance with higher number of nodes in our experiments: the system becomes notably slower with more than approx. 10 clients joining the network.
Additionally, we have observed that inconsistencies occurred using the Firefox version of the Zeroties addon after approx. 10 clients joining the network and sending a significant amount of messages.
This was also the reason for performing our performance measurements only with Google Chrome.
While, further evaluation with Firefox after fixing these issues would harden our results, we do not think our sole focus on Chrome for now constitutes a significant threat.
But these issues certainly need some inspection and debugging before Zeroties can be deployed for larger and more critical applications.

\textbf{Deployability.}
What we have implemented as the \textit{Zeroties daemon} was originally intended to reside in the browser addon.
We could not find a way to achieve this due to the security restrictions imposed on addons by browsers (i.e. they can start neither a HTTP server nor a WebSocket server).
This is why Zeroties users will now have to start the OS-level (i.e. Node.js application) daemon to make use of our system.
We intend to research further possibilities to go without this separate dependency in the future.
For example, there may be a more advanced possibility for security configuration in browsers that we were not able to find yet.

\textbf{Usability and network requirements.}
Different characteristics of Zeroties in its current version may distract users (or even make our system entirely unusable for them). 
First, we are using hard-coded ports for our HTTP and WebSocket servers.
While one might think initially that this could simply be solved by providing configuration options, it is actually not that simple: depending on the number of applications, Zeroties may need multiple HTTP and WebSocket servers that must be started and stopped dynamically.
Nevertheless, such configuration options are required because, as obvious, hard-coded port numbers are not feasible in practice.
Another problem in terms of usability are the URLs of Zeroties servives that remain IP-based.
This has already been a limitation in Successorships and we could not yet overcome this problem.
Finally, we have not yet implemented filtering of DNS-SD services to only include Zeroties services in the lists of available services.
In practice, this means that, for example, a Google Chromecast service may currently appear in the Zeroties menu.
This is a simple fix using an internal prefix for Zeroties service keys and respective filtering of all DNS-SD services.

\textbf{Network load.}
Each Zeroties daemon currently polls the network router for a list of available Zeroties services.
While we see this approach mostly as justified in our specific target environments (see Section~\ref{sub:design}), it has been hard for us to estimate how much load our system will impose on the local network.
Currently we have configured our daemon to poll the service list in intervals of one second.
With each host participating in a Zeroties network performing this polling strategy, there is a risk of significant impact on the network.
We intend to evaluate this in more detail with a future evaluation of Zeroties.






\section{Related Work}
\label{sec:related_work}

\cite{hong2007accelerating}

\section{Conclusion}
\label{sec:conclusion}

In this paper, we presented Zeroties, an asynchronous publish-subscribe service for web applications operating in local ad-hoc networks. 
Our evaluation has shown that Zeroties makes our previously developed framework Successorships available on a variety of browsers and speeds up failure recovery by almost a factor of 10.
While we can still see lots of potential for improvement in terms of robustness, scalability, and expressiveness, we think that we are now one large step closer to such applications becoming reality.
With Zeroties, we hope to contribute to more leveraging of local network infrastructure and a decrease in overall Internet communication because we think that physical and geological location have not been taken into account enough in the recent developments of the Internet.

\bibliographystyle{abbrv}
\bibliography{paper}

\end{document}


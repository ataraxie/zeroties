\subsection{Assumptions}
\label{sub:assumptions}

We make some strongly simplifying \textit{assumptions} in the first version of \APIName to make initial design and implementation tractable. In particular, we assume the following, being aware of necessary improvements for a productive version of our implementation.

\textit{Assumption 1: The nodes in the network trust one another.}
An implication of our model is that any node in the network (with a reliable connection) may at some point become the application server.
This status comes with all the responsibilities of the server, including running the server logic, and hence maintaining the server-side application state.
Without any further substantial design considerations, this model would be extremely susceptible to a malicious client acquiring server status, and exploiting this status for their own ends, either by serving downright malicious material, or by more subtly forging application state.

\textit{Assumption 2: Updates to server-side state are small.}
A core requirement of our model is that the server be able to broadcast its changes in state to all clients in the local network with relatively low latency, to diminish the possibility of clients' copies of server state being out of sync.
Therefore, we assume that communication in the network is not traffic intensive, i.e. nodes neither produce bursts of requests in a small period of time, nor do they have payloads higher than some threshold $\tau$ that might generate bottlenecks in the network.

\textit{Assumption 3: Applications are restricted to a client-server model with one state object.}
\APIName apps follow a client-server model and are built using one shared object. While this may sound very limiting at first, we can see this becoming a norm in modern Web applications, especially with the popularity of frontend frameworks like React\footnote{https://reactjs.org/} and Redux\footnote{https://redux.js.org/docs/introduction/}. In these frameworks, any change in the application's frontend is triggered by an operation on a single state object. We posit that \APIName aligns very well with these frameworks and consider this design choice well-justified.
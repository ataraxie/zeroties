\subsection{Availability and fault tolerance}
\label{sub:background_replication}


One strategy of replication, \textit{lazy replication}\footnote{also called \textit{optimistic replication}} \cite{Ladin:1990,Ladin:1992}, aims to provide the highest possible performance and availability by sacrificing consistency significantly.
With this approach, replicas periodically exchange information, tolerating out-of-sync periods; thus, state may diverge on replicas, but is guaranteed to converge when the system quiesces for some period of time.
This underlying consistency model is termed ``eventual consistency'', and has recently come to prominence in such high-profile applications as online editing platforms, NoSQL cloud databases, and big data processing frameworks \footnote{\url{http://www.oracle.com/technetwork/consistency-explained-1659908.pdf}, accessed 2017-10-08}.
Eventual consistency is a weak consistency model, providing no guarantee for safety as long as replicas have not converged.
A stronger model of eventual consistency is offered by \textit{conflict-free replicated data types} (CRDTs) \cite{Shapiro:2011}: any two replicas that receive the same updates, no matter the order, will be in the same state.

Clearly, the choice of consistency model impacts the application designer, since the level and types of guarantees required vary depend a great deal on the specific application.

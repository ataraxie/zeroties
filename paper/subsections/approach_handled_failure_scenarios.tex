\subsection{Handled Failure Scenarios}
\label{sub:approach_handled_failure_scenarios}

With our replication and consistency strategies, we handle different scenarios of failures gracefully.

\textbf{Client disconnection.} 
While disconnecting clients are mostly a trivial circumstance in traditional Web apps, they are more complex in \APIshort apps because the system relies on clients being represented in the shared \texttt{successors} list. 
A disconnecting client will trigger a \texttt{close} event on the current server for the associated WebSocket channel and the disconnected client's ID is obtained. 
The ID is then removed from the \texttt{successors} list on the shared state and the update is broadcasted. 
If the disconnected client was the first element in the \texttt{successors} list, the list will be shifted and the second element will be the new successor.

\textbf{Server disconnection.} 
A disconnected server will result in a WebSocket \texttt{close} event triggered at each connected client. 
After some time, the service discovery mechanism described in Section~\ref{sub:approach_conceptual_description} will reveal that there is no more service currently registered with the app's service name. 
This is the time when one of the previously connected client nodes will become the next server\footnote{We considered starting the next server immediately subsequent to a WebSocket close event but restrained from this approach, considering it too error-prone. For example, a simple page reload on the current server would trigger a WebSocket \texttt{close} event on all clients and one of them immediately becoming the next server. This would introduce another race condition between the first successor and the original server that registers itself again immediately after page reload. Our existing approach deals with this problem much more gracefully: the original server will simply remain the server because the set of current services is updated much less frequently. The downside of our approach is a significantly increased time of recovery.}: each client compares the first item in the shared \texttt{successors} list with its ID (obtained with the \texttt{welcome} message of the initial handshake). 
If it matches, the client will publish a server as described in Section~\ref{sub:approach_conceptual_description}. 
Otherwise, it will connect as soon as a service with the app's service name is available in a subsequent service discovery event.

\textbf{Simultaneous disconnection.} 
In this scenario, the server and one or multiple clients fail within a short time frame. 
For example, both the server and the first successor could disconnect and other clients would not receive the update that the first successor disconnected. 
If we did not target this case specifically, no client would become the new server because they all expect a disconnected client to do so. 
We solve this problem with a \textit{repeated timer} algorithm: clients that are not the first item in the \texttt{successor} list set a timer that will prune the first item from the list after some time. 
This interval is set sufficiently to give the current first successor enough time to publish the service. 
The interval is varied with a random number to make it highly unlikely that two clients prune successors at about the same time which could result in race conditions publishing a service. 
At some point in this process, one other client will be the first in the \texttt{successors} list and will become a server. 
Other clients will be notified eventually by a service discovery event and connect to the new server.